\section*{ВВЕДЕНИЕ}
\addcontentsline{toc}{section}{ВВЕДЕНИЕ}
По данным журнала Market Research Report \cite{statistics} рынок видеостриминга сейчас оценивается более, чем 500 миллиардов долларов, и, предполагается, что к 2030 году эта сумма достигнет 1.9 триллионов долларов. Согласно статистике по всему миру насчитывается около 1.8 миллиарда подписок на сервисы потоковой передачи видео, примерно 26\% пользователей которых признаются, что пользуются подпиской на постоянной основе не реже одного раза в неделю. 

Потоковая передача видео в настоящее время более популярна и составляет более, чем 38.1\% от общего объёма использования, чем кабельное или широковещательное телевидение, на долю которых приходится 30.9\% и 24.7\%.

Соответственно, ввиду непрекращающегося спроса на видеоплатформы, необходимо обеспечить эффективный способ взаимодействия многочисленных пользователей с этими сервисами для обеспечения наиболее качественной передачи информации.

Цель работы -- разработать протокол для мультисерверного стриминга, обеспечивающий высокую производительность и масштабируемость.

Для достижения цели необходимо решить следующие задачи:
\begin{itemize}	
	\item проанализировать предметную область и выделить целевую аудиторию;
	
	\item определить сценарии взаимодействия участников передачи данных, составить набор соответствующих сообщений и их содержание;
	
	\item провести обзор существующих форматов видеорядов и обосновать выбор используемого формата в разрабатываемом протоколе;
	
	\item идентифицировать и сформулировать основные требования для мультисерверного стриминга;
	
	\item разработать протокол в соответствии с выделенным и требованиями;
	
	\item реализовать и протестировать прототип, позволяющий проверить правильность работы и эффективность протокола.
\end{itemize}

\pagebreak