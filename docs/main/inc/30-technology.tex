\section{Технологическая часть}
\subsection{Выбор средств программной реализации}
\subsubsection{Основные средства}
В качестве языка программирования был выбран Python 3 \cite{python}, ввиду нескольких причин.
\begin{itemize}
	\item Также язык поддерживает объектно-ориентированный подход, что важно, поскольку в процессе реализации подразумевается использование этой методологии, позволяющей разрабатывать хорошо организованную \, и \, просто модифицируемую структуру приложения.
	\item Кроме того, предоставляются библиотеки для создания графического интерфейса, которые планируется использовать для отладки и наглядной демонстрации работы приложения.
	\item В дополнение, в процессе обучения был накоплен существенный опыт в использовании этого языка программирования. \newline
\end{itemize}
%
В качестве среды разработки был выбран VSCode \cite{vscode} в силу следующих факторов.
\begin{itemize}
	\item Она бесплатна для студентов.
	\item Предоставляются удобные инструменты для написания, редактирования кода, а также графический отладчик.
	\item Помимо этого, является хорошо знакомой средой разработки, и какие-либо проблемы с взаимодействием сведены к минимуму, что позволяет сэкономить время.
\end{itemize}

\subsubsection{Вспомогательные средства}
Для сбора датасета использовалась кроссплатформенная система мгновенного обмена сообщениями Discord \cite{discord}. Это было сделано по следующим причинам.
\begin{itemize}
	\item Она бесплатна для всех пользователей.
	\item Поскольку необходимо было опросить большое количество людей, встречаться лично было затруднительно, поэтому гораздо удобнее и проще организовать весь процесс дистанционно, что и позволяет сделать это приложение.
	\item Платформа очень популярна среди молодых людей, что доказывает недавнее исследование от апреля 2022 года \cite{discordS}. Это упрощает поиск удобной большинству среды. 
	\item Для этой системы возможно написание дополнительного API в \, виде \, discord-бота, который может выполнять различные задачи. Так, для ускорения процесса фиксации слов опрашиваемых было создано дополнительное приложение, которое в реальном времени переводит речь участников в текст, заносит всю информацию в файлы и по каждому человеку создаёт базу знаний.
\end{itemize}

\subsection{Используемые библиотеки}
Для разработки графического пользовательского интерфейса привлекалась библиотека PyQt5 \cite{pyqt5}. Qt -- один из самых популярных кроссплатформенных графических фреймворков \cite{qt}, поэтому кроме документации, описано множество примеров его использования. Для наглядной разработки GUI привлекалась среда Qt Designer \cite{qtdesigner}.

Для упрощения контроля над тем, правильно ли формируются графы и сети, используется библиотека NetworkX \cite{networkx}, позволяющая визуализировать подобные структуры. 

В приложении и discord-боте для перевода речи пользователя в текст используется библиотека Speech Recognition \cite{speech_rec}. Это инструмент от таких компаний, как Google, Microsoft, IBM и др. Для работы используется стандартный Google Speech API.

К другой библиотеке, Natasha \cite{natasha}, происходит обращение с целью выделения словосочетаний в предложениях. Natasha позволяет  с помощью готовых правил решать базовые задачи NLP для русского языка такие, как:
\begin{itemize}
	\item токенизация;
	\item сегментация;
	\item определение морфологических признаков;
	\item лемматизация/нормализация;
	\item выделение словосочетаний и т.д.
\end{itemize}

Кроме того, привлекается библиотека SciPy \cite{scipy}, которая ориентирована на работу с большим количеством данных, содержит много функций линейной алгебры, интерполяции, масштабирования данных. \newline

\subsection*{Выводы}
%\addcontentsline{toc}{subsection}{Выводы}
В данном разделе для реализации разрабатываемого \, метода \, выбран \, Python в качестве основного языка программирования, Javascript -- как вспомогательный инструмент для реализации сопутствующего приложения для сбора датасета. В качестве среды разработки был выбран PyCharm.

Определены основные используемые библиотеки. Также изложен способ получения данных для статистического метода.

Кроме того, приведён и подробно описан интерфейс программы и продемонстрирована её работа. А также изложены основные пункты, по которым производилось тестирование ПО.
\pagebreak