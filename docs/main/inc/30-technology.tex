\section{Технологическая часть}
\subsection{Выбор средств программной реализации}
\subsubsection{Основные средства}
В качестве языка программирования был выбран Python 3 \cite{python}, ввиду нескольких причин.
\begin{itemize}
	\item Язык поддерживает объектно-ориентированный подход, что важно, поскольку в процессе реализации подразумевается использование этой методологии, позволяющей разрабатывать хорошо организованную \, и \, просто модифицируемую структуру приложения.
	\item Кроме того, предоставляются библиотеки для создания графического интерфейса, которые планируется использовать для отладки и наглядной демонстрации работы приложения.
	\item В дополнение, в процессе обучения был накоплен существенный опыт в использовании этого языка программирования. \newline
\end{itemize}
%
В качестве среды разработки был выбран VS Code \cite{vscode} в силу следующих факторов.
\begin{itemize}
	\item Бесплатна.
	\item Предоставляются удобные инструменты для написания, редактирования кода, а также графический отладчик.
	\item Помимо этого, является хорошо знакомой средой разработки, и какие-либо проблемы с взаимодействием сведены к минимуму, что позволяет сэкономить время. \\
\end{itemize}

\subsection{Используемые библиотеки}
\textbf{OpenCV} -- open source библиотека компьютерного зрения, которая применяется для анализа, классификации и обработки изображений и видео. \cite{cv2}

\textbf{gRPC} -- библиотека, которая используется для обеспечения взаимодействия между клиентом и мастер-сервером по проколу gRPC. \cite{grpc} \\

\subsection*{Выводы}
%\addcontentsline{toc}{subsection}{Выводы}
В данном разделе для реализации протокола в качестве основного языка программирования был выбран Python, среды разработки -- VS Code. Определены основные библиотеки, в том числе и для обработки видеоматериалов. 

\pagebreak