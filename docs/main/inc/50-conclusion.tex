\section*{ЗАКЛЮЧЕНИЕ}
\addcontentsline{toc}{section}{ЗАКЛЮЧЕНИЕ}
Таким образом, в рамках текущей научно-исследовательской работы было определено понятие SYN-атаки, её особенности. Также был сделан обзор существующих методов смягчения. На основе этого анализа можно сделать следующие выводы.
\begin{itemize}
	\item В процессе установки соединения создаются такие структуры ядра, как: struct sk\_buff, struct request\_sock, struct sock, struct tcp\_timewait\_sock, struct inet\_request\_sock, struct inet\_sock, struct inet\_connection\_sock и некоторые другие.
	
	\item Атаки такого вида направлены на переполнение SYN-очереди, ведущей к затормаживанию и неработоспособности всей системы в целом, поскольку при заполненной очереди сервер не может принимать никакие пакеты, в том числе и ACK на уже принятые SYN-запросы.
	
	\item Одним из наиболее распространённых подходов к смягчению SYN-атак является комплексный метод на основе заранее заданных правил, с помощью которых можно вручную отслеживать параметры сети, и при необходимости обосновать почему произошёл тот или иной сигнал тревоги, также привлекается динамическое изменение параметров системы: время ожидания и размер очереди.
\end{itemize}
